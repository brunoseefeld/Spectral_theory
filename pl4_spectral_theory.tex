\documentclass{article}
\usepackage{graphicx} % Required for inserting images
\usepackage{amsmath}
\usepackage{amsfonts}




\title{Problem set 4 Spectral Theory}
\author{Bruno Seefeld}
\date{May 2024}

\begin{document}

\maketitle





\section{}
\subsection*{(1)}
Take $v$ in the ideal I, by polar decomposition we have $v=u|v|$, with $|v|$ positive, we can write $|v|=u^* v$, this implies $|v|=|v|^\ast \in I$, therefore 
$v^\ast=|v|^\ast u^*\in I$.

\subsection*{(2)}

Let $I=\{fz|f\in C(\mathbb{Z})\}$ where $z$ is the identity on $\mathbb{D}$. We prove that $z^\ast$ that does $\mapsto \overline{z}$ is not in $I$. Suppose 
there is $f\in I$ with $fz=z^\ast$, for $x\in \mathbb{R}$ we have $f(x)x=x$ therefore $f(x)=1$, and $f(ix)(ix)=-ix$ and $f(ix)=-1$, taking $x\to 0$ we see that $f$ is 
not continuous at $0$, a contradiction. Therefore the ideal is not self adjoint.  



\section{}


\section*{5}

 \quad\quad   Suppose $p,q$ are Murray-von Neumann equivalent, with $p=vv^\ast$, $q=v^\ast v$. Let $v=u|v|$ be the polar
decomposition of $v$, with $u$ unitary and $|v|$ self adjoint. We have $p=p^2=u|v|^2u^\ast$ 
and $q=|v|u\ast u |v|=|v|^2$
clearly $p=uqu^\ast$.

Now suppose $p,q$, $1-p,1-q$ are unitarlly equivalent. Say $1-p=u(1-q)u^\ast=uu^\ast -uqu^\ast$, we have
$p=uqu^\ast$ so we can suppose these pairs are conjugated by the same unitary transformation. 
Let's calculate $p=p^2=(uqu^\ast)(uqu^\ast)=(uq)(qu^\ast)=(uq)(uq)^\ast $, take $v=uq$. we have $p=vv^\ast$
and $v^\ast v= qu^\ast u q=q$, therefore they are Murray-von Neumann equivalent.




\end{document}







