\documentclass{article}
\usepackage{graphicx} % Required for inserting images
\usepackage{amsmath}
\usepackage{amsfonts}




\title{Problem set 4 Spectral Theory}
\author{Bruno Seefeld}
\date{May 2024}

\begin{document}

\maketitle





\section{1}
\subsection*{1}
Take $v$ in the ideal I, by polar decomposition we have $v=u|v|$, with $|v|$ positive, we can write $|v|=u^* v$, this implies $|v|=|v|^\ast \in I$, therefore 
$v^\ast=|v|^\ast u^*\in I$.

\subsection*{2}

Let $I=\{fz|f\in C(\mathbb{Z})\}$ where $z$ is the identity on $\mathbb{D}$. We prove that $z^\ast$ that does $\mapsto \overline{z}$ is not in $I$. Suppose 
there is $f\in I$ with $fz=z^\ast$, for $x\in \mathbb{R}$ we have $f(x)x=x$ therefore $f(x)=1$, and $f(ix)(ix)=-ix$ and $f(ix)=-1$, taking $x\to 0$ we see that $f$ is 
not continuous at $0$, a contradiction. Therefore the ideal is not self adjoint.  



\section{2}


\end{document}







