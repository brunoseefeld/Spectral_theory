\documentclass{article}
\usepackage{graphicx} % Required for inserting images
\usepackage{amsmath}
\usepackage{amsfonts}
\usepackage[]{dsfont}




\title{Problem set 4 Spectral Theory}
\author{Bruno Seefeld}
\date{May 2024}

\begin{document}

\maketitle





\section*{1}
\subsection*{(1)}
Take $v$ in the ideal I, by polar decomposition we have $v=u|v|$, with $|v|$ positive, we can write $|v|=u^* v$, this implies $|v|=|v|^\ast \in I$, therefore 
$v^\ast=|v|^\ast u^*\in I$.

\subsection*{(2)}

Let $I=\{fz|f\in C(\mathbb{Z})\}$ where $z$ is the identity on $\mathbb{D}$. We prove that $z^\ast$ that does $\mapsto \overline{z}$ is not in $I$. Suppose 
there is $f\in I$ with $fz=z^\ast$, for $x\in \mathbb{R}$ we have $f(x)x=x$ therefore $f(x)=1$, and $f(ix)(ix)=-ix$ and $f(ix)=-1$, taking $x\to 0$ we see that $f$ is 
not continuous at $0$, a contradiction. Therefore the ideal is not self adjoint.  


\section*{2}
\subsection*{1)}
The fact that this is an algebra follows from the fact that $\ell_\infty$ and
$\mathcal{K}(\ell_2)$ are algebras. The same goes for it being a $*$-algebra.
It has the $C^*$ property becauase $\|(a+k)*(a+k)\|=\|a+k\|^2$, just consider
the fact that $a+k\in B(\ell_2)$ and this is a $C^\ast$-algebra.
We only need to show that it's Banach. But a Cauchy sequence in $\ell_\infty+
\mathcal{K}(\ell_2)$ is a Cauchy sequence in $B(\ell_2)$, a Banach space, so 
it converges, we only need to show that the limit is in $\ell_\infty+
\mathcal{K}(\ell_2)$.


Take $(y_n)_\subset \ell_\infty + \mathcal{K}(\ell_2)$ a sequence converginc
to $y\in B(\ell_2)$. Define 
by

\begin{align*}
    T:B(\ell_2)\to & \ell_\infty \subset B(\ell_2) \\
    Ta(n)=\langle ae_n,e_n  \rangle 
\end{align*}

clearly $Tx=x$ for $x\in \ell_\infty$. Routine arguments show
that $T$ is linear, we have:

\begin{equation}
    \|T\|=\sup_{\|b\|\leq 1} \|Tb\|_\infty=\sup_{\|b\|\leq 1} |\langle
        be_n,e_n
    \rangle|\leq \|b\|=1
\end{equation}
 so it's also continuous.

 Now $y_n=a_n+ b_n\to_n y$. Writing $y=\Phi(y)-(\Phi(y)-y)$, in order to show
 that $y\in  \ell_\infty+ \mathcal{K}(\ell_2)$ we need to show 
$\Phi(y)-y$ is compact, but

\begin{equation}
    \Phi(y)-y=\lim_{n} \Phi(a_n)+\Phi(b_n)-a_n-b_n=\lim_n \Phi(b_n)-b_n
\end{equation}

since $b_n$ is compact we need to show $\Phi(b_n)$ is compact, equivalently,
show that $\langle b_n e_k,e_k \rangle\to_k 0 $. This follows considering by 
taking a sequence $c_k\to_k b_n$ of finite rank operators and considering that
each $c_k$ is a linear combination of rank one projections, for a rank one
projection $\alpha$, $\lim_k \langle \alpha e_k,e)k \rangle =0$. Hence $\Phi(b_n)$
is compact and $\ell_\infty + \mathcal{K}(\ell_2)$. 
\subsection*{2)}
We prove first that the image of a rank 1 projection is taken to a rank 
1 projection. For this we use that if $p,q$ are projections such that 
$q\leq p$, than the following holds:

\begin{align*}
    pq=qp&=q\\
    q(\ell_2)\subseteq & p(\ell_2)
\end{align*}

so rank 1 projections are minimal with respect to the partial order $\leq$.
Let $p$ be a rank 1 projection and suppose by contradiction that $\Phi(p)$ 
is not minimal, then there exists a projection $q$ with $q\leq \Phi(p)$ and
$q\neq \Phi(p)$. We know by minimality that $\Phi(p)q=q$, so $p\Phi^{-1}(q)=
\Phi^{-1}(q)$, therefore $p=\Phi^{-1}(q)$ and $\Phi(p)=q$, a contradiction.

Since $\mathcal{F}(\ell_2)$ is spanned by rank 1 projections, 
$\Phi(\mathcal{F}(\ell_2))\subseteq \mathcal{F}(\ell_2) $, the inverse
inequality is obtained by the same argument for $\Phi^{-1}$. 
Since $\Phi$ is continuous and $\mathcal{F}(\ell_2)$ is dense in $\mathcal{K}
(\ell_2)$, we have $\Phi(\mathcal{K}
(\ell_2))=\Phi(\overline{\mathcal{F}(\ell_2)})=\overline{\Phi(\mathcal{F}(\ell_2))}
\subseteq \mathcal{K}
(\ell_2)$. We get equality by applying the same thing for $\Phi^{-1}$, which
is also continuous. 




\section*{3}
\subsection*{1}
Suppose $\sigma(u)$ has only one point $\alpha\in \mathbb{C}$. Then by the spectral theorem we have 
$u=\int z dE=\alpha I$, since the dimension is bigger than 1, we can find
a non trivial closed subspace $V$ and $uV=\alpha I V=V$.
If $\sigma(u)$ contains more than one point we take two non-empty disjoint Borel subsets 
$\omega,\omega'$ of $\sigma(u)$ and consider $E(\omega),E(\omega')$, we have
that $uE(\omega)=E(\omega)u$ because if $v$ commutes with $u$ and $u*$
then $v$ commutes with $f(u)$ for $f\in B_{\infty}(\sigma(u))$. Let $M=\text
{Range} E(\omega), M'=\text
{Range} E(\omega')$, if $x\in M$, $uE(\omega)x=ux=E(\omega)ux$, therefore $M$
is invariant by $u$. If $x\in M$, since $E(\omega\cap\omega')=E(\omega)E(\omega')=0$
$E(\omega')x=0$, therefore $x\in (M')^\perp$, if $y\in (M')^\perp$, then
$y=E(\omega')y+E(\omega)y=E(\omega)y$, so that $y\in M$, hence $M=(M')^\perp$. 
So $M$ is a non trivial closed invariant vector subspace.

\subsection*{2}

We start proving the following: for $\lambda\in \sigma(u)$, Null$(u-\lambda)=
\text{Range}(E(\{\lambda\}))$. 

To prove this recall that for $f\in C(\sigma(u))$, Null$(f(u))=\text{Range}
E(f^{-1}(0))$, consider $f(z)=z-\lambda$, we have Null$(u-\lambda)=\text{Range}
(\{\lambda\})$. 

Now, since the spectral measure is regular and $\lambda$ is an isolated point, 
$\{\lambda\}$ is an open set, therefore $E(\{\lambda\})\neq 0$, therefore by
the previous result we have Null$(u-\lambda)\neq 0$ and $\lambda$ is an eigenvalue.


For completness let's prove that Null$(f(u))=\text{Range}(E(f^{-1}(0)))$. 

Take a continuous $f:\sigma(u)\to \mathbb{C}$ and let $\omega=f^{-1}(0)$. 
We have $f(z)\mathds{1}_{\omega}=0$ and by the exended functional calculus we 
have $f(u)E(\omega)=0$, then if $x\in\text{Range}E(\omega)$, $f(u)E(\omega)x=
f(u)x=0$, there is, $x\in\text{Null}(f(u))$. 

Now let $\omega_n=\{z\in \sigma(u)|\frac{1}{n+1}\leq|f(z)|< \frac{1}{n}\}$
and \[f_n(z)=
\begin{cases}
    \frac{1}{f(z)} & \text{if}\quad z\in \omega_n \\
    0 & \quad \text{otherwise}
\end{cases}\]

with this we get $f_n(z)f(z)=\mathds{1}_{\omega_n}$. Since $\sigma\setminus
\omega=\sqcup_{n=1}^{\infty}\omega_n$ and $E$ is countably additive we have
$E(\omega)+\sum_n E(\omega_n)=I$, so if $x\in \text{Null}f(u)$, $E(\omega_n)x
=0$, hence $E(\omega)x=x$, so that $x\in \text{Range}(E(\omega))$.








\section*{5}

 \quad\quad   Suppose $p,q$ are Murray-von Neumann equivalent, with $p=vv^\ast$, $q=v^\ast v$. Let $v=u|v|$ be the polar
decomposition of $v$, with $u$ unitary and $|v|$ self adjoint. We have $p=p^2=u|v|^2u^\ast$ 
and $q=|v|u\ast u |v|=|v|^2$
clearly $p=uqu^\ast$.

Now suppose $p,q$, $1-p,1-q$ are unitarlly equivalent. Say $1-p=u(1-q)u^\ast=uu^\ast -uqu^\ast$, we have
$p=uqu^\ast$ so we can suppose these pairs are conjugated by the same unitary transformation. 
Let's calculate $p=p^2=(uqu^\ast)(uqu^\ast)=(uq)(qu^\ast)=(uq)(uq)^\ast $, take $v=uq$. we have $p=vv^\ast$
and $v^\ast v= qu^\ast u q=q$, therefore they are Murray-von Neumann equivalent.


\section*{6}

Suppose $p,q$ are Murray-von Neumann equivalent, we have $p=uqu^\ast$ for $u$ unitary. Take $x\in\text{Range}(p)$, then $px=x$,
we have $u^\ast(x)=qu^\ast(x)$ so $u^\ast(x)\in \text{Range}(q)$, now take $y\in \text{Range}(q)$ 
$uy=uqy=puy$ having $y=u^\ast(puy)$, with all this we get $\text{Range}(q)=u^\ast \text{Range}p$.
By reversing the roles of $q$ and $p$ and $u$ and $u^\ast$ we have $\text{Range}p=
u\text{Range}q$. By using that $(1-p)=u(1-q)u^\ast$ we also have that $\text{Null} (p)=q\text{Null}(q)$
and $\text{Null}(q)=u^\ast\text{Null}(p)$. 


Now suppose $\text{Range}(p)=u\text{Range}(q)$ and $\text{Range}(q)=u^\ast \text{Range}(p)$ 
and $\text{Null}(p)=u\text{Null}(q)$, $\text{Null}(q)=u^\ast\text{Null}{p}$.
Take $x\in H$, we know from the above that $u^\ast px=qy$ for some $y$. We also know 
$qu^\ast(1-p)x=0$, then $qu^\ast x=qu^\ast px=qqy=qy$ therefore $u^\ast px=qu^\ast x$ and they 
are Murray-von Neumann equivalent.


So we have the following characterization of Murray-von Neumann equivalence: $p,q$ are 
Murray-von Neumann equivalent if and only if $\text{Range}(p)=u\text{Range}(q)$ and $\text{Range}(q)=u^\ast \text{Range}(p)$ 
and $\text{Null}(p)=u\text{Null}(q)$, $\text{Null}(q)=u^\ast\text{Null}(p)$.


\section*{7}
Take $\epsilon<\frac{1}{2}$. Define $x=pq+(1-p)(1-q)$. We have 
\[  
\|x-1\|=\|p(p-q)+(q-p)q\|<2\|p-q\|<1    
\]

So that $x$ is invertible. Take $x=uz$ the polar decomposition of $x$, we have
that $z$ is also invertible and $z^{-1}=u^\ast x$. By opening up the terms
that define $x$ we have $px=xq=pq$ and $qx^\ast x=qpq$, also $x^\ast x q=qpq$,
then we have $qz^2=z^2q$, but since $q^2=q$, $qz=zq$, finally:

\[pu=pxz^{-1}=xqz^{-1}=uzqz^{-1}=uqzz^{-1}=uq\]

then $u$ and $q$ are unitarily equaivalent. 




\section*{9}
\subsection*{1)}
Suppose $p$ and $q$ are homotopic. Then there is a continuous $\Phi:[0,1]\to A$ with $\Phi(0)=p$
$\Phi(1)=q$ and $\Phi(t)$ a projection for all $t$. For every $t\in [0,1]$, we can find $\delta_t$
such that $\|\phi(t)-\phi(t')\|<1$ for $|t'-t|<\delta_t$. Since the interval is compact
we can cover it with a finite number of intervals of the form $(t-\delta_t,t+\delta_t)$. 
Say $\{I_i\}_{i=1}^{N}$ is such a finite collection of intervals, by taking smaller $\delta_t$
we can make an arbitrary $x\in [0,1]$ intersect at most $2$ of those intervals. Also by
changing their names if necessary we can have $t_{i+1}>t_i$. So $p=\Phi(0)\in I_1$, hence
$p$ is unitarlly equivalent to everyone in $I_1$, take a point $x\in I_1\cap I_2$, we have 
that $\Phi(x)$ is unitarily equivalent to everyone in $\Phi(I_2)$, by transitivity $p$ is 
unitarily equivalent $\Phi(I_2)$. By continuing with taking points in the intersections 
$I_i\cap I_{i+1}$ we have that $p$ and $q$ are unitarily equivalent.  

\subsection*{2)}

We know that if $\|p-q\|<1$ then they are unitarily equivalent, there is, $p=uqu^\ast$ for 
some unitary $u$, write $u=e^{ia}$ for some self adjoint $a$. Take $\Phi=e^{ita}qe^{-ita}$.
For $t=0$ we have $\Phi(t)=q$ and $\Phi(1)=p$, also $\Phi(t)^2=e^{ita}qe^{-ita}e^{ita}qe^{-ita}=
e^{ita}qe^{-ita}=\Phi(t)$. We only need to show $\Phi$ is continuous: since the exponential
map is continuous, multiplication by $q$ is continuous, continuity of $\Phi$ follows. 
We have $p$ and $q$ homotopic.


\subsection*{3)}

Take $A=M_3(\mathbb{C})$ and $p(x,y,z)=(x,y,0)$ and $q(z,y,z)=(0,0,z)$.
If they were homotopic, they would be unitarily equivalent by item 1, but
then $\text{Range } p=u (\text{Range } q)$, which implies $p$ and $q$ have the 
same rank, a contradiction, therefore they are not homotopic.
\end{document}







