\documentclass{article}
\usepackage{graphicx} % Required for inserting images
\usepackage{amsmath}
\usepackage{amsfonts}




\title{Problem set 4 Spectral Theory}
\author{Bruno Seefeld}
\date{May 2024}

\begin{document}

\maketitle





\section{}
\subsection*{(1)}
Take $v$ in the ideal I, by polar decomposition we have $v=u|v|$, with $|v|$ positive, we can write $|v|=u^* v$, this implies $|v|=|v|^\ast \in I$, therefore 
$v^\ast=|v|^\ast u^*\in I$.

\subsection*{(2)}

Let $I=\{fz|f\in C(\mathbb{Z})\}$ where $z$ is the identity on $\mathbb{D}$. We prove that $z^\ast$ that does $\mapsto \overline{z}$ is not in $I$. Suppose 
there is $f\in I$ with $fz=z^\ast$, for $x\in \mathbb{R}$ we have $f(x)x=x$ therefore $f(x)=1$, and $f(ix)(ix)=-ix$ and $f(ix)=-1$, taking $x\to 0$ we see that $f$ is 
not continuous at $0$, a contradiction. Therefore the ideal is not self adjoint.  



\section{}


\section*{5}

 \quad\quad   Suppose $p,q$ are Murray-von Neumann equivalent, with $p=vv^\ast$, $q=v^\ast v$. Let $v=u|v|$ be the polar
decomposition of $v$, with $u$ unitary and $|v|$ self adjoint. We have $p=p^2=u|v|^2u^\ast$ 
and $q=|v|u\ast u |v|=|v|^2$
clearly $p=uqu^\ast$.

Now suppose $p,q$, $1-p,1-q$ are unitarlly equivalent. Say $1-p=u(1-q)u^\ast=uu^\ast -uqu^\ast$, we have
$p=uqu^\ast$ so we can suppose these pairs are conjugated by the same unitary transformation. 
Let's calculate $p=p^2=(uqu^\ast)(uqu^\ast)=(uq)(qu^\ast)=(uq)(uq)^\ast $, take $v=uq$. we have $p=vv^\ast$
and $v^\ast v= qu^\ast u q=q$, therefore they are Murray-von Neumann equivalent.


\section*{6}

Suppose $p,q$ are Murray-von Neumann equivalent, we have $p=uqu^\ast$ for $u$ unitary. Take $x\in\text{Range}(p)$, then $px=x$,
we have $u^\ast(x)=qu^\ast(x)$ so $u^\ast(x)\in \text{Range}(q)$, now take $y\in \text{Range}(q)$ 
$uy=uqy=puy$ having $y=u^\ast(puy)$, with all this we get $\text{Range}(q)=u^\ast \text{Range}p$.
By reversing the roles of $q$ and $p$ and $u$ and $u^\ast$ we have $\text{Range}p=
u\text{Range}q$. By using that $(1-p)=u(1-q)u^\ast$ we also have that $\text{Null} (p)=q\text{Null}(q)$
and $\text{Null}(q)=u^\ast\text{Null}(p)$. 


Now suppose $\text{Range}(p)=u\text{Range}(q)$ and $\text{Range}(q)=u^\ast \text{Range}(p)$ 
and $\text{Null}(p)=u\text{Null}(q)$, $\text{Null}(q)=u^\ast\text{Null}{p}$.
Take $x\in H$, we know from the above that $u^\ast px=qy$ for some $y$. We also know 
$qu^\ast(1-p)x=0$, then $qu^\ast x=qu^\ast px=qqy=qy$ therefore $u^\ast px=qu^\ast x$ and they 
are Murray-von Neumann equivalent.


So we have the following characterization of Murray-von Neumann equivalence: $p,q$ are 
Murray-von Neumann equivalent if and only if $\text{Range}(p)=u\text{Range}(q)$ and $\text{Range}(q)=u^\ast \text{Range}(p)$ 
and $\text{Null}(p)=u\text{Null}(q)$, $\text{Null}(q)=u^\ast\text{Null}(p)$.

\end{document}







