\documentclass{article}
\usepackage{graphicx} % Required for inserting images
\usepackage{amsmath}
\usepackage{amsfonts}
\usepackage[]{dsfont}




\title{Spectral theory problem set 6}
\author{Bruno Seefeld}
\date{June 2024}

\begin{document}

\maketitle

\section*{1)}

We start with the following: if $\Phi$ is positive and for every 
$a\in A^+, \|a\|\leq 1$ we have $\|\phi(a)\|<M$  for some $M\in \mathbb{R}^+$
then $\|\phi\|\leq 4M$. 
This follows by writing any $x\in A$ as $x=a_{sa}+ib_{sa}$ for self adjoint 
$a_{sa},b_{sa}$ and then $a_{sa}=a^+ -a^{-}, b_{sa}=b^+ -b^{-}$, then $x$
is the linear combination of $4$ positive elements, we have that $\|\phi\|\leq 4M$.

Notice that by the $C^\ast$ equality we can take the sup on the definition of
the norm of $\phi$ to be over positive elements since $\|a\|^2\leq 1\implies
\|a^\ast a\|\leq 1$. 

And finally to prove that $\phi$ is bounded, assume by contradiction that 
it isn't. Then there is a sequence of positive elements $(a_n)_n$ in the 
unit ball of $A^+$  with $\|\phi(a_n)\|\geq 2^n$. Define $a=\sum_{i=0}^\infty 
\frac{a_i}{2^i} $, we have for $N\in \mathbb{N}^+$: $\sum_{i=0}^{N-1} 
\frac{a_i}{2^i}\leq a$, hence $N\leq \|\sum_{i=0}^{N-1} 
\phi(\frac{a_i}{2^i})\|\leq \|\phi(a)\|$, a bound for all natural numbers, impossible.
Hence for positive elements, $\phi$ is bounded, overall boundedness follows from
the first observation. In this argument we use that if $x\in A^+$ then 
$x\leq \|x\|$, which may require us to work wih unitization.


\section*{2)}

Convexity: let $\mu\in (0,1)$,$\tau_1,\tau_2\in S(A)$ and $a\in A^+$, we have $\mu\tau_1(a)+(1-\mu)\tau_2(a)\geq 0$ since all this numbers are non-negative.

Weakly closed: let $(\tau_\lambda)_\lambda$ be a converging net and $\tau$ it's limit, take $a\in A^+$, $\tau(a)=\lim_\lambda \tau_\lambda(a)\geq 0$, so that $\tau$ is
positive. Take an approximate unity $(u_\beta)_\beta$, $\|\tau\|=\lim_\beta \tau(u_\beta)=\lim_\beta \lim_\lambda \tau_\lambda(u_\beta)=
\lim_\lambda\lim_\beta\tau_\lambda(u_\beta)=\lim_\lambda 1=1$, so $\tau\in S(A)$.



\section*{4)}

Suppose $a$ is invertible, take $x\in A$, we can write $x=a a^{-1}x a^{-1}a$,
so $x\in aAa$ and $A=aAa$.

Now suppose $a$ is stricly positive, since $A=aAa$ we can find $x$ such that
$\|1-axa\|<1$, hence $axa$ is invertible, this implies $a$ is invertible. 

\section*{5)}

Suppose $a$ does not have a dense range. Then there exists a finite dimensional
subspace $V\in H$ with $V\cap \text{range} a=\{0\}$. Let $p$ be the orthogonal
projection onto $V$, $p\in K(H)$, but if $x\in aAa$, then $\text{range} x\subseteq 
\text{range} a $, therefore $p\notin aAa$ and $a$ is not strictly positive.
We have shown $a$ strictly positive implies it have dense range.

Now suppose $a$ has dense range. Since the rank 1 projections span $K(H)$, 
we'll show that $x\otimes y \in aAa$. For $x,y\in H$ take sequences $(u_n)_n$
$(v_n)_n$ such that $\lim_n a(u_n)=x, \lim_n a(v_n)=y$, this is possible since
the range is dense.
Notice that  for $z\in H$

\begin{align*}
    (x\otimes y)(z)=& \langle z,\lim_n a(v_n) \rangle \lim_n a(u_n)=\\
    a (\langle az, \lim_n v_n \rangle \lim_n (u_n))=&\lim_n a(u_n\otimes v_n)a(z)\in aK(H)a
\end{align*}

so $K(H)\subseteq aK(H)a$ and $a$ is strictly positive.


\section*{6)}

Since our algebra is unital, $a$ being stricly positive implies it's invertible.
Hence if $(\mathcal{H},\phi)$ is the universal representation, we have that $\phi(a)$ is
invertible, in particular injective. 
Let's assume by contradiction that there exists a positive functional
 $\tau$ that $\tau(a)=0$, we can normalize it if needed and consider $\tau$
 a state. 
 Since $a$ is positive, there exists self adjoint and invertible  $b$ with $a=b^* b$,
 so $\tau(b*b)=0$, which implies $\tau(bx)=0$ for all $x\in H$. In particular
 the class of $b$ in the ideal $H/N_\tau$ is $0$. Take a non zero $u$ in 
 the Hilbert space completion of $H/N_\tau$, and let $(y_\lambda)_\lambda \in 
 \mathcal{H}$ with $y_\lambda=u $ if $\lambda=\tau$ and $0$ otherwise.
 We have that $\phi(b)(y_\lambda)_\lambda=0$, a contradiction since $b$
 is invertible and $\phi$ is a $*$-isomorphism. 
 We have that for all positive functionals, $\tau(a)>0$.  


 \section*{7)}

 If $A$ has an approximate unity $(u_n)_n$ define $a=\sum_{i=0}^\infty 
 \frac{u_n}{2^n}$. Let $\phi$ be a positive functional, we can normalize if 
 needed and consider it a state. We have $\|\phi\|=1=\lim_n \phi (u_n)$,
 therefore after some $n_0$, $\phi(u_n)>0$, this implies $\phi(a)>0$, therefore
 a is a strictly positive element. Note: this is an equivallent definition
 of stricly positive, for completedness we prove it at the end.

 Suppos $a$ is a strictly positive element. We claim that $Aa$ is dense in A.
 In fact, suppose it is not, then $Aa$ forms a left ideal and there is a 
 state $\phi$ that vanishes on the closure of $Aa$, hence on $a$, a contradiction
 since $a$ is strictly positive. Threfore $Aa$ is dense. Now let $f$ be the 
 image of $a$ by functional calculus, we have that $ff(f+ \frac{1}{n})^{-1}\to f$,
 therefore $aa(a+\frac{1}{n})^{-1}\to a$. For $b\in A$, there exists by density
 a $c\in A$ with $\|b-ca\|<\frac{\epsilon}{3}$,

 \begin{align*}
    &\|b-ba(a+\frac{1}{n})^{-1}\|=  \\ &\|b-ca+ca -caa(a+\frac{1}{n})^{-1}
    +caa(a+\frac{1}{n})^{-1}-ba(a+\frac{1}{n})^{-1}\|\leq \\
    &\|b-ca\|+\|ca -caa(a+\frac{1}{n})^{-1}\|+ \|caa(a+\frac{1}{n})^{-1}-ba(a+\frac{1}{n})^{-1} \|\leq \\
    &\frac{\epsilon}{3}+ \frac{\epsilon}{3} + \|ca-b\|\leq \epsilon 
 \end{align*}



 \section*{8)}

 Let $S$ be he right shift on $\ell_2(\mathbb{Z})$, i.e, $S((a_n)_n)=a_{n+1}$.
 Consider $Y=\{(a_n)_n\in \ell_2(\mathbb{Z})|a_n=0\quad  \forall n>0\}$. This is 
 a subspace invariant by the shift. Take $e_0$ the canonical basis vector, $e_0\in Y$
 but $S^{-1}(e_0)= e_1\notin Y$, therefore $S$ is not invertible in $Y$ and
 $0\in \sigma(S_{|Y})$, but by an exercise on a previous problem set 
 $\sigma(S)=\mathbb{S}^1$, this implies $\sigma(S_{|Y})\not\subset \sigma(S)$.

















\end{document}







