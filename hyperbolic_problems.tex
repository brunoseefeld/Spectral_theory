\documentclass{article}
\usepackage{graphicx} % Required for inserting images
\usepackage{amsmath}
\usepackage{amsfonts}
\usepackage[]{dsfont}




\title{Problems in hyperbolic dynamics}
\author{Bruno Seefeld}
\date{June 2024}

\begin{document}

\maketitle

\begin{abstract}
Some problems that I'm strugling with

\end{abstract}

\section{}
If $\Lambda$ is an isolated hyperbolic set and $f_\Lambda $ is a transitive diffeomorphism, then $\overline{\text{Per}(f_\Lambda)}=\Lambda$. 
The $\Lambda\subseteq \overline{\text{Per}(f_{|\Lambda)}} $ follows from Anosov closing lemma. The other direction seems to follow from the fact that $\Lambda$ is closed.



Proof of the second part: 

Let $V$ be the neighborhood from the isolation of $\Lambda$. Suppose $x$ is 
accumulating periodic points of $f_{|\Lambda}$ and it's not in $\Lambda$. 
Then ther is $n\in \mathbb{Z}$ such that $x\notin f^n (V)$. We can, by 
reducing $V$ if necessary, assume $x\notin \overline{f^n(V)}$. Since this
set is open we have an $r>0$ such that $B(f^{n}x,r)\subset \overline{f^n(V)} $.
By the uniform continuity of $f$, we can take $\delta$ such that $d(u,v)<\delta$
implies $d(f^{-n}(u), f^{-n}v)<r$. Find $p_k$ periodic point $\delta$ close to 
$x$, we have that $f^{-n}(p_k)\notin f^{n}(\Lambda)$, a contradiction because
$\Lambda$ is isolated.










\end{document}







