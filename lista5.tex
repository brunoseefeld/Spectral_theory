\documentclass{article}
\usepackage{graphicx} % Required for inserting images
\usepackage{amsmath}
\usepackage{amsfonts}
\usepackage[]{dsfont}




\title{Problem set 5 Spectral Theory}
\author{Bruno Seefeld}
\date{May 2024}

\begin{document}

\maketitle


\section*{Problem 1}

First we have that $\sigma(uau^\ast)=\sigma(u^\ast ua)=\sigma(a)$.

Notice that $(uau^\ast)^n=ua^nu^\ast$, this implies that for a polynomial 
$p$, $p(uau^\ast)=up(a)^\ast$. So we have, for $x,y\in H$
\begin{align}
    \langle up(a)u^\ast x,y \rangle=&\langle p(uau^\ast )x,y \rangle\\
    \int_{\sigma(a)}p(z) dE_{u^\ast x^\ast y}=&\int_{\sigma(a)}p(z) dE'_{x,y}  
\end{align}

where $E,E'$ are the measures corresponding to $uau^\ast, a$ respectively.

Take a sequence of polynomials $p_n(z)$ converging pointwise to $e^{iz}$, we
have by dominated convergence theorem that 

\begin{equation}
    \int_{\sigma(a)} \lim p_n(z) dE_{u^\ast x^\ast y}=\int_{\sigma(a)}\lim p_n(z) dE'_{x,y} 
\end{equation}
which implies $\langle e^{iuau^\ast}x, y\rangle =\langle ue^{ia}u^\ast,y\rangle$ 
for all $x,y\in H$.
This of course making the possibly dangerous asasumption that $e^{it}$ is integrable,
it's certainly bounded since $\sigma(a)\subset \mathbb{R}$.




\section*{Problem 3}

\subsection*{1)}

Let $t,s\in \mathbb{R}$

\begin{align*}
    \sigma_{t+s}(a)=e^{i(t+s)h}ae^{-(t+s)h}=& e^{ith}e^{ish}ae^{-ith}e^{-ish}\\
    =e^{ith}e^{ish}ae^{-ish}e^{-ith}=&\sigma_{t}(\sigma_s (a))
\end{align*}

the commutation in the equality on the first line because $-ith,-ish$ commute.


\subsection*{2)}


No. Take for example $H=\ell_2(\mathbb{N})$ and $h$ the self-adjoint unbounded
operator given by $h(a_n)_n=(h_n a_n)_n$ where $(h_n)_n\subset \mathbb{R}$
is an anbounded sequence and $(a_n)_n\in \text{dom} h$.
By functional calculus provided by the spectral measure of $h$ we have 
$e^{ith}(a_n)_n=(e^{ith_n}a_n)_n$. Since the sequence is unbounded, we can find
for each $k\in \mathbb{N}$ a $n_k$ such that $|h_{n_k}-h_k|>k$. Let $E_nm$ be 
the rank one operaor that does only this $E_nm (e_n)=e_m$. 

We can write $e^{ith}=\sum_{k}e^{ith_k}E_{kk}$.  Define $a=\sum_k E_{k n_k}$,
for $x=\sum_k x_k e_k$, $\langle (I-a^\ast a x),x\rangle =\|x-\sum_{n_k} x_{n_k}e_{n_k}\|^2 \geq 0 $,
so $a^\ast a \leq I$ and $a\in B(H)$. 

Let's use all this to compute $\| e^{ith}ae^{-ith}-a      \|$:

\begin{align*}
    \| e^{ith}ae^{-ith}-a      \|=&\|e^{ith}\sum _k e^{-ith_k}E_{k n_k}-a  \|=\\
    \|\sum_k (e^{it(h_{n_k}-h_k)}-1)E_{k n_k} \|\geq & \sup_k |e^{it(h_{n_k}-h_k)}-1|
\end{align*}

take a sequence $t_k=\frac{\pi}{h_{n_k}-h_k}$, by our choice of $h_{n_k}$ 
we have $\lim_k t_k=0$ but the norm is bounded below by 2, so we don't have
strong continuity in this sense.





\section*{Problem 4}

Proving $(3)\iff (2)$, by $a\text{dom} h\subseteq \text{dom}(h) $ we can write 
$i\langle ax, hy\rangle-i\langle ahx, y\rangle=-\langle hax,y\rangle 
-i\langle ahx,y\rangle =\langle i[h,a]x,y\rangle$ so the sesquilinear form
is bounded iff $i[h,a]$ is.

Proving $(1)\implies (3)$:
suppose the derivative exists as a bounded operator in it's domain, then 
it's given by 
\begin{align}
    \frac{d}{dt} e^{ith}ae^{-ith}=ihe^{ith}ae^{-ith}-ie^{ith}ahe^{-ith}
\end{align}
when calculated at $t=0$ we have $iha-iah$, so $i[h,a]$ is a bounded operator 
in $\text{dom} h$.










\end{document}

